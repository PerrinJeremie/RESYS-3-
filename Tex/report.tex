% A LaTeX (non-official) template for ISAE projects reports
% Copyright (C) 2014 Damien Roque
% Version: 0.2
% Author: Damien Roque <damien.roque_AT_isae.fr>

\documentclass[a4paper,12pt]{book}
\usepackage[utf8]{inputenc}
\usepackage[T1]{fontenc}
%\usepackage[frenchb]{babel} % If you write in French
\usepackage[frenchb]{babel} % If you write in English
\usepackage{a4wide}
\usepackage{graphicx}
\graphicspath{{../Figures/}}
\usepackage{subfig}
\usepackage{tikz}
\usetikzlibrary{shapes,arrows}
\usepackage{pgfplots}
\pgfplotsset{compat=newest}
\pgfplotsset{plot coordinates/math parser=false}
\newlength\figureheight
\newlength\figurewidth
\pgfkeys{/pgf/number format/.cd,
	set decimal separator={,\!},
	1000 sep={\,},
}

\usepackage{caption}

\usepackage{ifthen}
\usepackage{ifpdf}
\ifpdf
\usepackage[pdftex]{hyperref}
\else
\usepackage{hyperref}
\fi
\usepackage{color}
\hypersetup{%
	colorlinks=true,
	linkcolor=black,
	citecolor=black,
	urlcolor=black}

\renewcommand{\baselinestretch}{1.05}
\usepackage{fancyhdr}
\pagestyle{fancy}
\fancyfoot{}
\fancyhead[LE,RO]{\bfseries\thepage}
\fancyhead[RE]{\bfseries\nouppercase{\leftmark}}
\fancyhead[LO]{\bfseries\nouppercase{\rightmark}}
\setlength{\headheight}{15pt}

\let\headruleORIG\headrule
\renewcommand{\headrule}{\color{black} \headruleORIG}
\renewcommand{\headrulewidth}{1.0pt}
\usepackage{colortbl}
\arrayrulecolor{black}

\fancypagestyle{plain}{
  \fancyhead{}
  \fancyfoot[C]{\thepage}
  \renewcommand{\headrulewidth}{0pt}
}

\makeatletter
\def\@textbottom{\vskip \z@ \@plus 1pt}
\let\@texttop\relax
\makeatother

\makeatletter
\def\cleardoublepage{\clearpage\if@twoside \ifodd\c@page\else%
  \hbox{}%
  \thispagestyle{empty}%
  \newpage%
  \if@twocolumn\hbox{}\newpage\fi\fi\fi}
\makeatother

\usepackage{amsthm}
\usepackage{amssymb,amsmath}
\usepackage{array}
\usepackage{bm}
\usepackage{multirow}
\usepackage[footnote]{acronym}
\usepackage{multicol}

\newcommand*{\SET}[1]  {\ensuremath{\mathbb{#1}}}
\newcommand*{\VEC}[1]  {\ensuremath{\boldsymbol{#1}}}
\newcommand*{\FAM}[1]  {\ensuremath{\boldsymbol{#1}}}
\newcommand*{\MAT}[1]  {\ensuremath{\boldsymbol{#1}}}
\newcommand*{\OP}[1]  {\ensuremath{\mathrm{#1}}}
\newcommand*{\NORM}[1]  {\ensuremath{\left\|#1\right\|}}
\newcommand*{\DPR}[2]  {\ensuremath{\left \langle #1,#2 \right \rangle}}
\newcommand*{\calbf}[1]  {\ensuremath{\boldsymbol{\mathcal{#1}}}}
\newcommand*{\shift}[1]  {\ensuremath{\boldsymbol{#1}}}

\newcommand*{\eme}[1] {$#1^{\mbox{\footnotesize{ème}}}$}

\newcommand{\eqdef}{\stackrel{\mathrm{def}}{=}}
\newcommand{\argmax}{\operatornamewithlimits{argmax}}
\newcommand{\argmin}{\operatornamewithlimits{argmin}}
\newcommand{\ud}{\, \mathrm{d}}
\newcommand{\vect}{\text{Vect}}
\newcommand{\sinc}{\ensuremath{\mathrm{sinc}}}
\newcommand{\esp}{\ensuremath{\mathbb{E}}}
\newcommand{\hilbert}{\ensuremath{\mathcal{H}}}
\newcommand{\fourier}{\ensuremath{\mathcal{F}}}
\newcommand{\sgn}{\text{sgn}}
\newcommand{\intTT}{\int_{-T}^{T}}
\newcommand{\intT}{\int_{-\frac{T}{2}}^{\frac{T}{2}}}
\newcommand{\intinf}{\int_{-\infty}^{+\infty}}
\newcommand{\Sh}{\ensuremath{\boldsymbol{S}}}
\newcommand{\C}{\SET{C}}
\newcommand{\R}{\SET{R}}
\newcommand{\Z}{\SET{Z}}
\newcommand{\N}{\SET{N}}
\newcommand{\K}{\SET{K}}
\newcommand{\reel}{\mathcal{R}}
\newcommand{\imag}{\mathcal{I}}
\newcommand{\cmnr}{c_{m,n}^\reel}
\newcommand{\cmni}{c_{m,n}^\imag}
\newcommand{\cnr}{c_{n}^\reel}
\newcommand{\cni}{c_{n}^\imag}
\newcommand{\tproto}{g}
\newcommand{\rproto}{\check{g}}
\newcommand{\LR}{\mathcal{L}_2(\SET{R})}
\newcommand{\LZ}{\ell_2(\SET{Z})}
\newcommand{\LZI}[1]{\ell_2(\SET{#1})}
\newcommand{\LZZ}{\ell_2(\SET{Z}^2)}
\newcommand{\diag}{\operatorname{diag}}
\newcommand{\noise}{z}
\newcommand{\Noise}{Z}
\newcommand{\filtnoise}{\zeta}
\newcommand{\tp}{g}
\newcommand{\rp}{\check{g}}
\newcommand{\TP}{G}
\newcommand{\RP}{\check{G}}
\newcommand{\dmin}{d_{\mathrm{min}}}
\newcommand{\Dmin}{D_{\mathrm{min}}}
\newcommand{\Image}{\ensuremath{\text{Im}}}
\newcommand{\Span}{\ensuremath{\text{Span}}}


\newtheoremstyle{break}
  {11pt}{11pt}%
  {\itshape}{}%
  {\bfseries}{}%
  {\newline}{}%
\theoremstyle{break}

%\theoremstyle{definition}
\newtheorem{definition}{Définition}[chapter]

%\theoremstyle{definition}
\newtheorem{theoreme}{Théorème}[chapter]
\newtheorem{lemme}{Lemme}[chapter]
\newtheorem{corollaire}{Corollaire}[chapter]
%\theoremstyle{remark}
\newtheorem{remarque}{Remarque}[chapter]
\newtheorem{remarques}{Remarques}[chapter]
%\theoremstyle{plain}
\newtheorem{propriete}{Propriétée}[chapter]
\newtheorem{exemple}{Exemple}[chapter]

\parskip=5pt
%\sloppy

\usepackage{pdfpages}

\usepackage{chngcntr}
\counterwithout{figure}{chapter}
\counterwithout{table}{chapter}
\counterwithout{theoreme}{chapter}
\counterwithout{remarques}{chapter}
\counterwithout{definition}{chapter}
\counterwithout{lemme}{chapter}
\counterwithout{corollaire}{chapter}

\usepackage{longtable}
\usepackage{wrapfig}

\usepackage{hyperref}

\begin{document}

%%%%%%%%%%%%%%%%%%
%%% First page %%%
%%%%%%%%%%%%%%%%%%

\begin{titlepage}
\begin{center}
	

{\large RESYS Project}\\[0.5cm]

% Title
\rule{\linewidth}{0.5mm} \\[0.4cm]
{ \huge \bfseries Differentiation of hematopoietic precursors in embryos\\[0.4cm] }
\rule{\linewidth}{0.5mm} \\[1.5cm]

\vspace{2cm}

% Author and supervisor
\noindent
\begin{minipage}{0.4\textwidth}
  \begin{flushleft} \large
    M.~Jeremie \textsc{Perrin} \\ \small \textit{M2 BIM student}\\ \small \textit{ENS Cachan}\\
  \end{flushleft}
\end{minipage}%
\begin{minipage}{0.4\textwidth}
  \begin{flushright} \large
    M.~Corbin \textsc{Hopper} \\ \small \textit{M2 MPRI student}\\
  \end{flushright}
\end{minipage}

\vfill

% Bottom of the page
{\large  2019 -- 2020}

\end{center}
\end{titlepage}

\section*{The Dataset}
\subsection*{The Early Hematopoietic process}
	The biological process we are studying is one of differentiation of multipotent precursor cells \textit{Hemangioblasts} into hematopoietic and endothelial cells alike. In the mouse embryo, there is emergence of blood cells at day 7 in the yolk sac and this is the marker of the beginning of the hematopoiesis. Both the Hematopoietic Stem Cells (HSC) and the Endothelial Stem Cells (EPC) originate from these multipotent precursor cells. The blood cells will be formed by differentiation of the HSCs while the vasculature (tissue of the blood vessels) will be formed through differentiation of the ESCs.

\subsection*{The Experiment}
	In the paper \textit{Decoding the Regulatory Network for Blood Development from Single-Cell Gene Expression Measurements}, single-cell gene expression data was used in order to infer the regulatory networks at work during the hematopoietic differenciation process. It is known that blood development
	initiates at gastrulation from mesodermal cells, which initially have the potential to form blood, endothelium and smooth muscle cells. They showed that single-cell analysis of a developing organ coupled with computational approaches can reveal the transcriptional programs that control organogenesis.\\
	In order to acquire the data necessary they sampled single-cells in \textit{in vivo} mice embryos. The cells were sampled from the mesoderm and their potential to differentiate into blood cells was asserted thanks to expression of Flk1 and Runx1 expression. The sampling was done at four distinct time points.\\
	Those four times points define groups of cells, which are not hemogeneous since the differentiation process is asynchronous. That is to say some cells begin their differentiation process earlier than others :
	\begin{itemize}
		\item[E7.00] At this time point the cells are labeled "PS"
		\item[E7.50] –––––––––––––––––––––––––––––––––– "NP"
		\item[E7.75] –––––––––––––––––––––––––––––––––– "HF"
		\item[E8.25] At this time points cells were categorized into two different set. Those cells which expressed \textit{GFP} were labeled "4SG" and where considered as putative blood cells while those that did not where labeled "4SFG" and considered as putative endothelial cells. 
	\end{itemize}
	At each time points gene expression of a set of genes was measured in each cell, these genes were selected by hand as they were known to play a role in the process. Forty-six genes were selected, out of those : four were housekeeping genes in order to assess the quelity of the measures. Nine were markers known to identify the different cell states and thirty-three were transcription known to play a role in the transcriptional program underlying the differentciation process.
\subsection*{Categorizing Genes}
	So as to be pertinent in our analysis we first need to categorize the genes into the different cell states they belong to. We use the literature to guide us in our task. In the article, the authors underlign groups of genes as being caracteristic of the two end states:
	\begin{itemize}
		\item[$\bullet$] For Hematopoietic Cells :
		\item[$\bullet$] For Endothelial Cells :
	\end{itemize}
	In order to support our categories we plotted the distribution of the mean levels of gene expression in the different stages :


\section{Their Network Inference}
\section{Using MIIC}
\section{}
	
\end{document}