
\documentclass[14pt,aspectratio=169]{beamer}
\usetheme{Goettingen} %other options incld Madrid, Marburg, Boadilla, Goettingen, Hannover
%\usepackage[margin=1in]{geometry}           
%\geometry{letterpaper}                 
%\usepackage{graphicx}
\usepackage{epstopdf}
\DeclareGraphicsRule{.tif}{png}{.png}{`convert #1 `dirname #1`/`basename #1 .tif`.png}
\usepackage{cite}
\usepackage{caption}
\usepackage{subcaption}

\usepackage[utf8]{inputenc}
\usepackage[T1]{fontenc}
%\usepackage{amsaddr}

\usepackage[english]{babel}
\usepackage{hyperref}
\usepackage{amssymb}

\setbeamertemplate{itemize items}[circle]
\definecolor{mahog}{rgb}{0,.4,.4} %{0, 0.6,0.6}, {0,.8,.4}, {.2,.8,.2}
\usecolortheme[named=mahog]{structure}


\usepackage{amsmath}
\DeclareMathOperator*{\argmax}{arg\,max}
\DeclareMathOperator*{\argmin}{arg\,min}
\DeclareMathOperator{\EX}{\mathbb{E}}

\let\bfseriesbis=\bfseries \def\bfseries{\sffamily\bfseriesbis}

\newcommand{\rfig}[1]{Figure~\ref{#1}}
\newenvironment{point}[1]%
{\subsection*{#1}}%
{}

\setlength{\parskip}{0.3\baselineskip}

\title{Hematopoietic Regulatory Network Inference}
%\subtitle{}
\author{Jeremie Perrin, Corbin Hopper}
\institute{ENS Cachan}
\date{\today}

\begin{document}

\begin{frame}
\titlepage
\end{frame}
 
 \begin{frame}
 	\frametitle{The Early Hematopoietic  Process}
 	\begin{columns}[T] % align columns
 		\begin{column}{.48\textwidth}
 			\begin{figure}
 				\includegraphics*[width=  \linewidth]{./images/hematopoieisis.jpg}
 			\end{figure}
 		\end{column}%
 		\hfill%
 		\begin{column}{.48\textwidth}
 			Pluripotent cells differentiate into:
 			\begin{itemize}
 				\item Endothelial Cells
 				\item Hematopoietic Cells
 			\end{itemize}
 		\end{column}%
 	\end{columns}
 \end{frame}
 
 \begin{frame}
 	\frametitle{The Experiment}
 	\begin{columns}[T] % align columns
 		\begin{column}{.48\textwidth}
 			\begin{figure}
 				\includegraphics*[width=  \linewidth]{./images/sampling.png}
 			\end{figure}
 			\begin{figure}
 				\includegraphics*[width=  \linewidth]{./images/numbers.png}
 			\end{figure}
 		\end{column}%
 		\hfill%
 		\begin{column}{.48\textwidth}
 			The data acquired:\\
 			\begin{itemize}
 				\item 3934 single cells
 				\item 46 genes
 				\item Binarized expression
 			\end{itemize}
 		\end{column}%
 	\end{columns}
 \end{frame}
 
 \begin{frame}
 	\frametitle{Categorizing genes}
 	\begin{columns}[T] % align columns
 		\begin{column}{.48\textwidth}
 			\begin{figure}
 				\includegraphics*[width=  \linewidth]{../Preliminary/genepca.png}
 			\end{figure}
 		\end{column}%
 		\hfill%
 		\begin{column}{.6\textwidth}
 			Steps :\\
 			\begin{itemize}
 				\item Each gene is a 5D vector.
 				\item Principal Component Analysis to reduce dimensions.
 				\item K-means as unsupervised clustering.
 			\end{itemize}
 			Genes grouped as expected (known cell-type specific genes are grouped together).
 		\end{column}%
 	\end{columns}
 \end{frame}
 

\begin{frame}
\frametitle{Entropy Inference: Approach}
\small
\begin{itemize}
\item Maximum Likelihood Like
\item Maximize mutual information $I(X;Y)$ of graph $G(V,E)$
\item But many random distributions have some small probability of being correlated
\end{itemize}

\begin{equation} \label{eq:1}
\argmax_{E \in G} \sum_{v \in V} I(v;e_{in}) - \EX[I(A;B)]
\end{equation}

\end{frame}

\begin{frame}

\frametitle{Entropy Inference: Approach}
\begin{itemize}
\item Optimization can consider in-edges to each node independently
\small
$$\argmax_{E \in G} \sum_{v \in V} I(v;e_{in}) - \EX[I(A;B)] = \sum_{v \in V} \argmax_{e_{in} \in E} I(v;e_{in}) - \EX[I(A;B)] $$
\item Optimize is independent of $p(v)$
$$ = \sum_{v \in V} \argmin_{e_{in} \in v} H(v|e_{in}) - \EX[H(A|B)]  $$
\item Minimize the uncertainty of each gene over the set of genes that regulate it 
\end{itemize}
\end{frame}


\begin{frame}
\frametitle{Entropy Inference: Approach}
How to simplify $\EX[H(A|B)]$  ?
\begin{itemize}
\item Let $a$ be a vector of length $n$, $b$ an $n$ by $m$ matrix 
\item Assume $a$ and columns of $b$ are from the same set of random distributions
\item As $n$ approaches infinity, the probability that all possible $2^m$ vectors are in $b$ approaches $1$, and so:
$$ \EX[H(A|B)] \approx \log2(m+1) - \log2(m) $$
\end{itemize}
\end{frame}


\begin{frame}
\frametitle{Entropy Inference: Algorithm}
\small
\fbox{%
    \parbox{\textwidth}{%
\textbf{Infer Graph}

Begin with a fully connected graph $G(V,E)$.

For each vertex $v \in V$:

\hspace{.5 cm} if $H(v) = 0$: remove v

For each vertex $v \in V$:

\hspace{.5 cm}  Infer Node (v)
}}

\fbox{%
    \parbox{\textwidth}{%
\textbf{Infer Node (v)}

Let w be the set of all predecessors of v, such that $(w_i,v) \in G$.

For each directed edge (u,v):

\hspace{.5 cm}  Let w\char`\\u be the set w, excluding vertex u.

\hspace{.5 cm} If $H(v|w\char`\\u) - H(v|w) \leq \EX[H(Y|X) - H(Y|X\char`\\x)]$ : \hspace{.2 cm} 

\hspace{1 cm} Remove edge (u,v)

\hspace{1 cm} Infer Node (v)
    }%
}
\end{frame}



\begin{frame}
\frametitle{Entropy Inference: Algorithm}
\small
$$ \sum_{v \in V} \argmin_{e_{in} \in v} H(v|e_{in}) - \EX[H(A|B)]  =  \sum_{v \in V} \argmin_{e_{in} \in v} H(v|e_{in}) - \log_2(\frac{m+1}{m}) $$
Remove Edge if:
\begin{align}
\begin{split}
H(v|e_1,...,e_{m-1}) -  H(v|e_1,...,e_m) &\leq \EX[{H(a|b_1,...,b_{m-1}) -  H(a|b_1,...,b_m)}] \\ \\
H(v|e_1,...,e_{m-1}) -  H(v|e_1,...,e_m)  &\leq \log_2(\frac{m^2}{m^2-1})  
\end{split}
\end{align}
\end{frame}

\begin{frame}
\frametitle{Entropy Inference: Algorithm}
Why top down and not bottom up?
\begin{itemize}
\item Let there be a set of edges that collectively reduce $H(Y|X)$, but individually do not
\item Example of XOR: $H(Y|x_1)=H(Y|x_2)=H(Y)$, but $H(Y|x_1,x_2)=0$
\item Top Down: removing any edge increases $H(Y|X)$, so none are removed
\item Bottom up: algorithm checks edges one at a time, adds none, and stops
\end{itemize}

\end{frame}


\begin{frame}
\frametitle{Entropy Inference: Performance}
\begin{itemize}
\item Perfect on asymetric binary problems such as $Y=(x_1\hspace{.1cm} NAND\hspace{.1cm} x_2\hspace{.1cm} NAND\hspace{.1cm} x_3)\hspace{.1cm} AND\hspace{.1cm} (x_4\hspace{.1cm} OR\hspace{.1cm} x_5)$
\item[]
\item Undirected for symetric binary problems such as $Y=x_1 \hspace{.1cm}XOR \hspace{.1cm}x_2$
\end{itemize}
\end{frame}

\begin{frame}
\frametitle{Entropy Inference: Performance}
\begin{itemize}
\item Terrible on benchmark tests (BNLearn: Lizards, Coronary, Asia)
\item[]
\item Not robust to noise, poor performance on noisy binary problems
\item[]
\item Threshold to remove edges may be too lenient for $n < 2^m$, since $\EX[H(A|B)]$ assumption would not hold
\end{itemize}
\end{frame}

\begin{frame}
\frametitle{Entropy Inference: Hematopoietic Data}
\begin{itemize}
\item 4 genes were removed since their entropy alone was 0 (housekeeping) 
\item Algorithm removes 7 more genes, reducing the total to 35
\item Resulting average in-degree and out-degree is 28
\item Since $n < 2^m$ entropy inference is too lenient and graph is too dense
\item Instead used as preprocessing technique, followed by MIIC
\end{itemize}
\end{frame}

% poss some frames in between

\begin{frame}
\frametitle{Inferred Hematopoietic Regulatory Network}

%2 examples one makes sense one doesn't

\begin{figure}
	\includegraphics*[height = 0.55\textheight]{../Datasets/miic-trimmed/miic-trimmed.png}
\end{figure}
\begin{itemize}
\item Cdh1
	\begin{itemize}
	\item High precursor expression levels
	\item Clustered as hematopoietic
	\item Biologically produces epithelial cadherin
	\end{itemize}
\end{itemize}
\end{frame}


\begin{frame}
\frametitle{Inferred Hematopoietic Regulatory Network}

\begin{figure}
	\includegraphics*[height = 0.55\textheight]{../Datasets/miic-trimmed/miic-trimmed.png}
\end{figure}
\begin{itemize}
\item Cdh5
	\begin{itemize}
	\item Clustered as endothelial
	\item Biologically linked to the process of endothelial development
	% necessary for endothelial polarity during embryonic development.
	\item Inferred hub with in-degree of 6 and an out-degree of 7
	\end{itemize}
\end{itemize}
\end{frame}

\begin{frame}
	\frametitle{Their Network Inference Method}
	 	\begin{columns}[T] % align columns
		\begin{column}{.48\textwidth}
			\begin{figure}
				\includegraphics*[width=  \linewidth]{../Biblio/images/pap_network.png}
			\end{figure}
		\end{column}%
		\hfill%
		\begin{column}{.48\textwidth}
			Their method :
			\begin{enumerate}
				\item Build state-transition graph
				\item Reduce problem to SAT
			\end{enumerate}
			Pros of their method :
			\begin{itemize}
				\item Explanatory
				\item General approach\\
				Given enough cells
			\end{itemize}
		\end{column}%
	\end{columns}
\end{frame}

\begin{frame}
	\frametitle{Open question}
	\center
	\large
	How does one integrate an assumption of dynamicity into MIIC ?
\end{frame}


\end{document}